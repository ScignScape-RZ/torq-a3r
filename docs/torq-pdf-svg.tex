
\vspace*{10pt}
{\semihr{0.26\textwidth}}
\vspace*{6pt}
%\sectnum{1.}
\hspace{2pt}\secttitle{{PDF/SVG Formats and Text Mining}}
\vspace*{4pt}
{\semihr{0.26\textwidth}}
\vspace*{4pt}


\p{Earlier comments about multimedia cross-referencing also apply to 
text documents and, by extension, digital text readers, such as 
\PDF{} viewers.  Renderers for \PDF{} files 
do not usually interoperate with multimedia software, 
so that for example a user could not in general click on a chemical 
formula referenced in a research paper and then have an option 
of viewing a corresponding molecular-\ThreeD{} model --- even if the 
research text and \PDB{} (Protein Data Bank) or 
\SDF{} (Structure-Data File) formats coexist in 
one overarching data set.  The \TorqAthreeR{} system 
incorporates a hybrid \PDF{}/\SVG{} format whose goal 
is to integrate \PDF{} viewers with multimedia 
components.  This format works somewhat by analogy to \GIS{} systems: consider 
a \PDF{} page as akin to a digital map at moderate-to-high zoom levels; 
then sentences are like city blocks or avenues, \PDF{} page 
coordinates are analogous to latitude/longitude, 
and sentence boundaries could be highlighted via 
geometric formations analogous to shape files.  
It is easy to direct readers toward a specific 
page in some other publication, but it is harder to 
create a reference to a specific \textit{paragraph}, or 
\textit{sentence}, or even the occurrence of one 
specific technical term of Named Entity.  The example of 
geotagging is useful as a point of contrast: latitude and longitude 
coordinates (or other \GIS{} parameters, such as so-called 
\XYZ{} triples based on a global Mercatur projection) can be 
specified with arbitrary precision, so it is possible with one 
consistent tuple of numbers to unambiguously refer to 
a precise point on (or at some elevation above or below) the 
Earth's surface.  Any kind of media (text, images, videos, etc.) is 
able therefore establish links to \GIS{} maps merely by geotagging 
specific entities.  The \GIS{} case is unusual, however, 
in that the entire geoinformatics field is almost entirely devoted 
to one specific spatial domain (viz., Earth's surface); most 
data spaces do not have comparable universal coordinate-systems.  
For example, we can cite one specific page within a text 
document (which is, perhaps, roughly equivalent to city or 
county, in the cartographic realm), but we cannot (in general) 
designate individual points \textit{within} a single text page.}

\p{Via a \PDF{}/\SVG{} hybrid, however, we can 
cite locations internal to a document page to the 
scale of fractions of a typographic 
point, similar to latitude and longitude coordinates 
on a map.  When displayed, \PDF{}/\SVG{} files thereby 
become interactive: regions such as page-areas spanned 
by sentences, paragraphs, or smaller text segments 
(such as the character-range within a technical term, 
keyword, acronym, or other Named Entity) can be 
assigned unique \SVG{} identifiers and take on 
distinct visual representations, as a cue to the 
user that these specific areas have special interactive 
possibilities.  For example, sentences might be highlighted 
when the mouse hovers on their interior, indicating that 
the sentence as a whole responds to mouse events; 
e.g., a context menu whose options might include 
copying the sentence text.  This is more 
convenient than the current convention, where 
text segments are copied by 
laboriously positioning the cursor at the 
first letter in a sentence and then dragging the 
mouse to the end (usually just beyond a closing 
period or other punctuation mark).  That kind of 
extended user action can be replaced with a single 
click, given proper text-encoding and cross-referencing 
between encoded text-segments and \PDF{} page coordinates.}

\p{Text mining and open data sharing  
have both emerged as key facets of contemporary 
scientific research, for related reasons: 
text mining and data transparency are both 
techniques for expanding the scope of individual 
research projects, maximizing the potential 
for insights and discoveries made possible by 
pooling multiple projects' data and methodology.  
However, existing research-oriented technologies reveal 
limitations that have prevented tech mining and 
data transparency from benefiting science to their 
full potential.  Key limitations include: 
flawed transcription of scientific text into machine-readable 
documents that are suitable for automated text-mining 
tools; lack of sufficient structure within published 
data to fully implement protocols related to 
microcitations, \q{nanopublications,} and other 
strategies for designating and tracking the provenance 
of fine-grained data aggregates within larger-scale 
research environments; and the absence of rigorous 
protocols for integrating and cross-referencing data 
and text across different media and genres, including 
scientific literature and raw data sets along with 
multimedia resources (videos, \ThreeD{} models, 
\GIS{} maps, image-series, etc.).}

\p{The limitations of existing scientific-publishing technologies, 
in particular, became especially evident during the \smbf{Covid19} 
pandemic, when scientists sought to unify 
researcher across many distinct disciplines --- from 
epidemiology and virology to clinical data, and biomolecular 
analysis of the \smbf{SARSCOVTwo} infection mechanism.  
A good example of the interdisciplinary response to the 
early pandemic was the \CORDNineteen{} corpus (\q{\smbf{Covid19} Open Research Dataset}), 
curated by the Allen Institute for Artificial Intelligence.  
This project made thousands of papers about Coronavirus biology 
freely available, but also compiled structured versions of all 
articles into a machine-readable format, encouraging programmers 
to develop text-mining and/or \AI{}-based tools for 
analyzing the corpus in its entirety.  This project actually helped to 
spotlight limitations of existing publishing technology: as we 
reviewed in a 2022 book about \smbf{Covid19} (as well as cancer and cardiac 
bioinformatics\footnote{\textit{See} \href{https://shop.elsevier.com/books/innovative-data-integration-and-conceptual-space-modeling-for-covid-cancer-and-cardiac-care/neustein/978-0-323-85197-8}{https://shop.elsevier.com/books/innovative-data-integration-and-conceptual-space-modeling-for-covid-cancer-and-cardiac-care/neustein/978-0-323-85197-8}}) 
the \CORDNineteen{} encoding was far from perfect, with 
annotation errors and inconsistencies between documents.  
For example, Named Entities such as chemical compounds were not 
correctly isolated from one document to the next, which led to missed 
opportunities for cross-referencing.  The Allen Institute acknowledged 
from the outset that their text extraction capabilities were limited 
by the formats (primarily \PDF{}) in which papers were presented, 
and literally issued a \q{call to action} requesting publishers to 
provide better-quality machine-readable encodings of text 
documents.\footnote{\q{CORD-19: The Covid-19 Open Research Dataset} 
(\dhref{https://www.ncbi.nlm.nih.gov/pmc/articles/PMC7251955/}), Section 5.2.}
We developed a more rigorous 
text-encoding system for our book about \smbf{Covid19} cited above, 
but in general the situation as of 2023 has not changed in the 
intervening three years.}

\p{Anti-patterns in publishing technology  
going beyond text mining alone: microcitation 
and cross-referencing practices --- especially when multiple genre and 
presentation/visualization media are involved --- remain 
haphazard and often unstandardized.  A comprehensive scientific and/or research corpus will likely 
encompass assets of diverse genres beyond just text documents: video, \ThreeD{}, \GIS{}, and so 
forth, as well as domain-specific data sets with their 
own media/visualization conventions.  In many cases there is 
no standard system for cross-referencing resources from 
distinct media --- even when there are obvious overlaps between 
content expressed in one media/genre and another, such as a 
chemical compound named (via molecular formulae) in a scientific 
text and also visualized in a 3-dimensional model.  Or, consider 
videos documenting the ecological effects of climate change, 
which could be cross-referenced against \GIS{} maps (via 
latitude and longitude coordinates).  In a cross-disciplinary 
context, many different forms of media and visuals can be relevant; 
\smbf{Covid19} research for instance made extensive use of \GIS{} models  
(tracking infectivity rates and related epidemiological data across 
the world) and also \ThreeD{} models (of the virus's notorious 
\q{spike proteins} in particular), diagnostic images (such as 
\q{ground-glass opacity} in lung scans) and electron-microscope 
pictures of the virus itself.  Despite the thematic 
correlation between these different media, however, the distinct 
software components which can show and analyze files and data sets 
of these various categories are often non-interoperable. 
Part of the problem is that data and file formats specific 
to individual media genre cannot necessarily be annotated with 
cross-references to other genres; but an equally significant 
limitation is that software is not equipped to identify and/or 
properly utilize such cross-references even when they are available. }

\p{The problems within text corpora and multimedia data sets are 
interrelated, or at least stem from similar causes: in each 
case there is ambiguity with respect to the individuation and 
description of particular information/presentation units,  
limiting the extent to which disparate software 
components can interoperate --- in particular,  
leverage shared data elements as a bridge from one component 
to another.  This is an underlying issue when attempting to  
define microcitations mutually recognized 
by variegated multimedia technologies.  
Consider again the case of geotagging video frames: 
restricting \GIS{} coordinates to a specific geographic 
area is one way to isolate specific parts of a \GIS{} database 
(or any data set with \GIS{} parameters as one 
data attribute, which might emerge from disciplines 
as diverse as epidemiology, sociodemographics, environmental 
science, ethology, anthropology, etc.).  Likewise, 
referring to one segment within a video represents a 
potential form of microcitation (as opposed to the video 
as a whole).  A microcitation protocol in these two 
domains would also allow application-level cross-references, 
so that software components dedicated to showing maps and 
videos, respectively, would at least have representations 
of the underlying data links permitting \GIS{} locale 
to serve as a bridge between video and geospatial media.  
However, \textit{absent} a common protocol the 
respective software components (digital map engines and 
video players) cannot interoperate.}


\p{The analogous problem in the domain of text-annotations derives 
from inconsistent standards with respect to textual citations and 
references more fine-grained than traditional bibliographic and page 
references.  As mentioned earlier, existing citation protocols 
do not incorporate text-spans or spatial locations 
more detailed than individual  
pages (except perhaps for specialized disciplines 
where natural-language texts are themselves objects of 
study; consider the practice of numbering individual 
lines of verse, within literary analysis).  With the 
growing importance of text-mining and scientific corpus analysis, 
the digital granularity conventionally associated with 
text encoding in linguistic and humanistic contexts should 
be extended to academic/scientific works in general.  This entails:

\begin{enumerate}[leftmargin=2pt, itemsep=-1pt]

\item{} A consistent schema for encoding natural-language texts as 
character streams, with proper disambiguation between glyphs with 
similar or identical appearance but playing different discursive/semantic 
roles (e.g., end-of-sentence periods from abbreviation punctuation 
and decimal points).

\item{} Openly declared conventions employed by each document with 
respect to the hierarchical nesting between paragraphs, sentences, 
smaller Named-Entity-like units, and other discursive/intratextual 
scales.  In other words, documents should clarify (as part of their 
encoding, markup, and/or metadata) such issues as sentence-boundaries 
in cases like embedded footnotes, transitions between paragraph 
and list-style formats, and other rhetorical devices where text 
deviates from the typical pattern of sentences following each other 
in sequence.  Likewise, text content that lies outside the 
normal paragraph/sentence/word hierarchy (e.g., phrases appearing within 
figure illustrations) should be encoded with its own character-sequence contexts.  
  
\item{} A \q{bibliospatial} coordinate system that can specify zero- and 
two-dimensional points/regions within the spatial extent of individual document 
pages to arbitrary levels of precision.  For example, an 
instruction such as \q{highlight the first sentence of the second full paragraph 
on page 3} should be read as referencing a clearly delineated 
geometric extent (typically a rectilinear octagon) on one 
visible (canonically \PDF{}) page.    

\item{} A suite of mathematical algorithms addressing numeric 
properties of bibliospatial systems --- e.g., second-order 
rectilinear Voronoi tessellation (related to the problem 
of mapping mouse/gesture events to the nearest bibliospatial object, 
such as a sentence, proximate to an event-location). 

\item{} A full mapping between bibliospatial coordinates and encoded 
machine-readable text: suppose we construe text encoding in terms 
of character-stream sets, so that text segments are located via 
(what we might call) \q{intertextual} coordinates.  Localized character 
streams (that could be cited, copied, cross-referenced, and so forth) 
could then be demarcated via start/end pairs sampled from one or 
more larger character streams.  Interop between \textit{machine-readable} 
and \textit{human-readable} versions of any document is then necessary 
to fully leverage bibliospatial capabilities; in other 
words, a full encoding should include mappings between 
intratextual and bibliospatial coordinates.  

\item{} A protocol for text-to-dataset and text-to-multimedia 
cross-referencing: that is, text segments marked via both 
bibliospatial and intratextual coordinates should be 
formatted in a consistent manner to be cited within computer 
code and multimedia assets so as to notate that a particular 
\q{subcontent,} or some region smaller than a multimedia file/resource 
in its entirety (e.g., a stream of video frames, a set of 
images within a series, a location visible within one configuration 
of a 360\textdegree{} photography viewer, etc.) is thematically 
linked to a region of text.   

\end{enumerate}
}

\vspace*{-6pt}
\p{When transitioning from individual documents to larger corpora 
or textual libraries/collections, a common encoding schema also becomes 
significant with respect to search queries and information retrieval.  The 
canonical situation where these concerns come into play is that of 
finding documents within a collection which match a query search-term.  
More sophisticated search engines try to consider not only the 
\textit{presence} of text snippets that match a search-term, but 
also the \textit{role} of the term in a matching document 
--- for example, a query intended to locate papers \q{about} 
\smbf{Covid19} would expect results prioritizing documents that 
not only contain the character-string \q{\smbf{Covid19}} somewhere, 
but also are thematically and semantically \textit{focused on} 
\smbf{Covid19} (in contrast to texts that might mention the 
pandemic but are mostly about other subjects).    
Distinguishing actual semantic relevance from \q{false positive} 
matches (documents which contain a search term but without 
containing much information \textit{relevant to} the search) 
is a separate matter from large-scale corpora queries, however, 
so for the current discussion we'll consider merely the problem 
of locating documents that do contain a particular query-term 
(which itself is non-trivial).  The overarching problem is that --- 
although we intuitively understand natural language as a single-dimensional 
flow of words and sentences --- the actual data through which 
documents are composed, stored, and rendered (into human-readable formats, 
such as \PDF{} pages) are more complex than simple character streams.}



\p{By way of illustration, suppose someone searches a list 
of publications for instances of a query term such as \q{Covid19} 
infectivity rate.  The assumption behind such a search is that 
some documents will contain a string of characters matching the 
search term (perhaps allowing for small variations, such as 
treating \q{Covid} and \q{COVID} as the same word, or 
matching \q{infection} as well as \q{infectivity}).  This is a 
reasonable assumption in the sense that as people read documents 
they subconsciously form a mental image of character-symbols 
connecting to form words, so it is usually obvious for \textit{humans} 
whether a given segment of text includes a given keyword or keyphrase 
as a proper part.  However, the ways that documents are internally stored 
can distort the \q{natural} string of characters, with markup and 
typographical details causing text segments to be interrupted or 
fragmented.  The difficulties mentioned earlier \visavis{} 
the \CORDNineteen{} corpus were due in part to formats such as 
\PDF{} introducing artifacts that distort each document's underlying 
character sequences.  A fully machine-readable text encoding 
would need to be a \textit{separate} resource from human-readable 
documents, one which unambiguously defines a canonical character stream 
covering the entire document, separate and apart from markup or 
presentation details.  For instance, the fact that certain characters 
are typeset in italics or boldface should not alter how they are 
matched against search queries (although, in some contexts, 
font choices such as italicization can be relevant --- suppose italics 
are used to state the title of a book; in this case a search 
might prioritize keywords that do appear within \textit{titles}, 
so this particular typographic convention should be taken into account).  
In effect, machine-readable encoding should permit the full 
scope of textual content to be queried as a single character stream, 
while also --- separate and apart from the stream itself (the 
separation of markup from content is often called \q{standoff annotation}) 
--- presenting information about text segments' typesetting and discursive 
details when relevant (i.e., which are book 
titles, Named Entities, technical terms, and other information that 
might come into play for search queries).}


\p{In sum, data- and resource-management for Translational Informatics naturally 
extends to natural language content-curation and text mining, so that software 
components dedicated to text documents should be intrinsic to \TSI{}-related technology.  As with other 
facets of Translational Research, computational techniques pioneered in biomedical and bioinformatics 
areas could certainly be extended to other scientific fields, particularly those related to 
community development and environmental sustainability.}




